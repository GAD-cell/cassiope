\begin{abstract}
%Recently much work has been devoted to developing GNNs beyond 1-WL in terms of expressivity. Many of these works achieve this on the trade-off of efficiency or limit the expressive superiority to specific types of graphs.
In recent years, there has been a significant amount of research focused on expanding the expressivity of Graph Neural Networks (GNNs) beyond the Weisfeiler-Lehman (1-WL) framework. 
While many of these studies have yielded advancements in expressivity, they have frequently come at the expense of decreased efficiency or have been restricted to specific types of graphs.
%In this paper, we explore GNN expressivity from the perspective of graph search. We propose a new node colouring scheme and show that under the scheme classical search algorithm can efficiently compute graph representation that goes beyond 1-WL. 
In this study, we investigate the expressivity of GNNs from the perspective of graph search. Specifically, we propose a new vertex colouring scheme and demonstrate that classical search algorithms can efficiently compute graph representations that extend beyond the 1-WL. 
We show the colouring scheme inherits useful properties from graph search that can help solve problems like graph biconnectivity.
Furthermore, we show that under certain conditions, the expressivity of GNNs increases hierarchically with the radius of the search neighbourhood.
%We develop a new type of GNN based on the scheme.  We show the differences between the two main search strategies and show what graph properties they can capture on top of 1-WL. We discuss techniques to further improve efficiency and show they can be applied to inductive settings.
To further investigate the proposed scheme, we develop a new type of GNN based on two search strategies, breadth-first search and depth-first search, highlighting the graph properties they can capture on top of 1-WL. Our code is available at \url{https://github.com/seanli3/lvc}.
\end{abstract}